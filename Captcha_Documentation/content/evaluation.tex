\section{Evaluation}
\label{sec:evaluation}

As proven by reCAPTCHA, the solving algorithm is accurate most of the time. However, there remains a small chance that images are labeled incorrectly. Certainly there is no way to completely eliminate the possibility of inaccurate labeled images. It would be possible to minimise this probability by further increasing the number of needed proposals, but this would slow the labeling process down and therefore result in fewer labeled images over time. The current implementation provides an acceptable trade-off in order to offer satisfying results for researchers, who rely on this service for data labeling.

Another major feature of the CaptchaService is the ability to reliably combat bots and spammers. The security features, which are in place right now will be enough to hold off the majority of all attacks. However one could argue, that it is not sufficient to renew the Captcha after a wrong proposal and validate the solved session of a user. Further security mechanism, such as blocking of ip addresses after consecutive inaccurate solutions could further increase the security of the system. Nevertheless, it will also worsen the user experience of clients that fail to solve Captchas reliably. This could ultimately lead to users not visiting the website again. Therefore the current implementation does not rely on any further security mechanisms and thus offers an optimal trade-off between user experience an security.

\clearpage

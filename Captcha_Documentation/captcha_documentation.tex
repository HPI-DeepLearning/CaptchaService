\documentclass[%
a4paper,
DIV12, 
2.5headlines, 
bigheadings, 
titlepage, 
openbib,
%draft
]{scrartcl}

%%% PACKAGES
\usepackage[ngerman, english]{babel}
%% FONTS


\usepackage[T1]{fontenc}
\usepackage{geometry}
\usepackage[utf8]{inputenc}
\usepackage{mathpazo}
\usepackage{helvet}
\usepackage{courier}
\usepackage{eurosym}
\usepackage{amsmath}
\usepackage{courier}
\usepackage{scrpage2}
\usepackage{graphicx}
\usepackage{xcolor}
\usepackage{multirow}
\usepackage{varioref}
\usepackage{babelbib}
\usepackage{makeidx}
\usepackage{tabularx}
\usepackage{floatflt}
\usepackage[pdftex, colorlinks, linktocpage, linkcolor=black, citecolor=black, urlcolor=black]{hyperref}
\usepackage[linesnumbered]{algorithm2e}
\usepackage{float}
\pagestyle{scrheadings}


\geometry{a4paper, top=55mm, left=40mm, right=35mm, bottom=40mm,
headsep=10mm, footskip=22mm}
\linespread {1.25}
%%% COMMANDS

	%%%%%%%%%%%%%%%%%%
	% Autor eintragen 
	\newcommand{\theauthor}{Marvin Gorecki Alexander Kromer Hendrik R{\"a}tz Robert Stark}
	%%%%%%%%%%%%%%%%%%
	% Titel eintragen 
	\newcommand{\thetitle}{Captcha Service}
	\newcommand{\thesubtitle}{Documentation}

%%% COLORS
\input{utils/hpicolors}

%%% OTHER INPUTS
\input{utils/commands}
\input{utils/environments}
\newcommand{\frontmatter}{\pagenumbering{roman}}
\newcommand{\mainmatter}{\pagenumbering{arabic}\setcounter{page}{1}}
%%% INCLUDE ONLY
\setlength{\parindent}{0cm}
\setlength{\parskip}{0.25cm}
%%% DOCUMENT
\begin{document}
	%%% HEADER AND FOOTTITLES
	%\selectlanguage{ngerman}
	\selectlanguage{english} % {ngerman}
	\automark{section}
	\ohead{\includegraphics[height=1.3cm,clip,viewport={0 60 250 180}]{utils/hpi_logo.pdf}}
	\chead{}
	\ihead{\headmark}
	\setheadsepline{1.0pt}[\color{hpigrey}]
	%%% TITLEPAGE
	\hypersetup{%
		pdftitle	= {\thetitle},
		pdfsubject	= {Captcha Documentation},
		pdfauthor	= {\theauthor},
		pdfcreator	= {PDFLaTeX},
		pdfproducer	= {LaTeX with hyperref and thumbpdf}
	}
	\titlehead{
%\parbox[b]{10cm}{\sffamily{\Large Hasso Plattner Institut}  \\Prof.~Dr.~Helmertstra�e~2-3 \\14482 Potsdam} 
\centering
\includegraphics[height=4cm]{utils/hpi_logo_text.pdf}

}
\title{\thetitle}
\subtitle{\thesubtitle}
%\author{{\small by}\\\textbf{\theauthor}}
%\dedication{Widmung\\mit mehreren\\Zeilen.}
\date{Potsdam, February 2017}
\publishers{
	\textbf{Supervisor}\\
	\vskip1em
	Prof. Dr. Christoph Meinel,\\		
	Christian Bartz\\

	\vskip2em
	\textbf{Internet-Technologies and Systems Group}
}
\frontmatter
\maketitle
	
	%%% Abstract
	\myabstract{%
	% deutsche Zusammenfassung
	}{%
	% englischer abstract
	The increasing numbers of bots, especially crawlers, within the World Wide Web has been a major concern for several years now. Over the years, different approaches to tackle bots were implemented and tested. Nowadays, one of the most widespread solutions for dealing with bots are Captchas. Originally, the main idea was to give users specific tasks to solve, which bots would be unable to solve. Through distortion and other obstacles, Captchas were improved against algorithmic solutions.
	
	The potential of million online users solving Captchas was quickly noticed. Difficulties in identifying words or objects in images using computers, could be solved using the combined solutions of Captcha users. We implemented our own Captcha Service in order to allow researchers and scientists to get their own datasets labeled using on	line users.
	}
	
	%%% TOC
	\tableofcontents
	\clearpage
	%%% INCLUDES
	\mainmatter

	%%%%%%%
	%% Add content here !!! %%%
	\section{Motivation}
\label{sec:motivation}

Machine and deep learning are great tools for tackling more and more complex tasks and problems. However, therefore a great number of data is required in order to train classifiers. Researchers and scientists are lacking the time and capacities to label data on their own, which is often further needed to advance and improve other technologies.

Using required authentication processes for online users, we are able to utilize huge amounts of free labor. Our main goal was building a straightforward service for researchers and scientists to allow precise data labeling. A Captcha service is an ideal solution for data labeling, while at the same time providing additional benefit by providing security from bot attacks.
We had to find a trade-off between a convenient, secure solution and a way to label data reliably.
A main criteria was also the expandability of the system, allowing to easily integrate further Captcha types for various data.

In order to allow a user friendly usage of the service, a web interface was required for uploading and downloading data.

Last but not least, a simple integration for web services was needed to allow for a fast and widespread usage of the Captcha service.

\clearpage

	\section{Related Work}
\label{sec:related_work}

http://vision.ucsd.edu/sites/default/files/soylentgrid.pdf

There are several Captcha types, each one being suitable for labeling different data.

Google's reCapcha is the major player for text Captchas. While google was analyzing  and making digital copies of books, their CRM algorithms were not able to identify every word. Google used reCaptcha to identify these words. The user was asked to identify two words, of which one word was known and used for validation.

*Bild von reCaptcha*

	\clearpage
	\section{Architecture}
\label{sec:architecture}

The architecture is mainly represented in the models.py data. It is designed for simple expandability and uses inheritance to simplify the introduction of new captcha types.  An overview is given in the class diageramm in figure \ref{fig:classdia}. 

\begin{figure}[!h]
\centering
\includegraphics[width=1.5\linewidth]{content/figures/classdiagramm.png}
\caption{Class diagramm representing the classes used for the generation of captchas. TODO text, update
}
\label{fig:classdia}
\end{figure}

It consists of two main classes, the \emph{CaptchaToken} and \emph{CaptchaSession}. The class \emph{CaptchaToken} represents a single image, that is part for a captcha, e.g. a single word, that needs to be written down by the user in order to solve the captcha. The class \emph{CaptchaSession} represents a complete captcha challenge a user has to solve, e.g. writting down the words shown on all images. Each type of captcha challenge provided by the service is represented by a subclass of \emph{CaptchaSession} and \emph{CaptchaToken}. Currently two kinds of Captchas, ImageCaptchas and TextCaptchas are supported.


All that needs to be done for implementing a new type of captcha challenge is to create a new subclass for \emph{CaptchaToken} and \emph{CaptchaSession} and implement specific functionality in these subclasses. Which methods and attributes need to be added in the new subclasses is listed in the ``Attributes and Methods implemented in the subclass''-paragraph.


All instances of a \emph{CaptchaToken} or \emph{CaptchaSession} are saved in the \verb|db.sqlite3|-Database. 

\subsection{CaptchaToken}
TODO konsistenz kuze beschr u einleitung
The class \emph{CaptchaToken} represents a single image, that is part for a captcha, e.g. a single word, that needs to be written down by the user in order to solve the captcha.

\paragraph{Attributes and Methods implemented in the superclass} \mbox{} \\

Attributes:

\begin{itemize}
\item \verb|file|: Image, that is represented by the CaptchaToken.
\item \verb|captcha_type|: String, that defines the type of captcha the token can be used for. Currently ``text'' for Textcaptchas and ``image'' for Imagecaptchas are supported.
\item \verb|resolved|: Boolean, that indicates, if the solution for a \emph{CaptchaToken} is known or not. A \verb|0| means the token is unsolved and a \verb|1| means the Token is solved.
\item \verb|proposals|: Dicitionary, that stores the possible solutions suggested by users of the captcha service and how often each solution was suggested.
\item \verb|insolvable|: Boolean that indicates, that a token is not solvable by clients of the captcha service.
\end{itemize}

Methods:

\begin{itemize}
\item \verb|create( file_name, file_data, resolved)|: Responsible for basic configuration, that need to be done for all kinds of tokens, when they are created. Only used for supercalls in the \verb|create()|-method of subclasses.
\item \verb|add_proposals(proposal)|: Adds a new suggested solution to the \verb|proposals|-dictionary, or increments the counter for an allready suggested proposal.
\end{itemize}

\paragraph{Attributes and Methods implemented in the subclass}  \mbox{} \\



Attributes: 


\begin{itemize}
\item \verb|result|: Saves the correct solution for a token. Datatype differs between different subclasses, e.g. \emph{TextCaptchaToken} saves a string and \emph{ImageCaptchaToken} saves a boolean.
\end{itemize}


Methods:


\begin{itemize}
\item \verb|create(file_name, file_data, resolved, result, insolvable=False)|: Resposible for configurating all attributes of the \emph{CaptchaToken}. Returns a \emph{CaptchaToken}.
\item \verb|try_solve|: Responsible for finding the correct solution for a \emph{CaptchaToken} based on the values saved in the \verb|proposals|-attribute.
\end{itemize}


\subsection{CaptchaSession}
TODO konsistenz kuze beschr u einleitung
Represents an instance of a captcha challange, that needs to be solved by a certain client. A \emph{CaptchaSessio}n consists of multiple ImageTokens, that are chosen randomly in order to create different challanges dynamically. Each Sessions corresponds to one of the supported types of \emph{CaptchaTokens}.


\paragraph{Attributes and Methods implemented in the superclass} \mbox{} \\


Attributes:

\begin{itemize}
\item \verb|session_key|: String, that serves as primary key to identify each session. 
\item \verb|origin|: String, that holds the IP adress that requested the captcha challenge. It is used match requests made by the client to the corresponding session.
\item \verb|session_type|: String, that defines the kind of captcha challenge, the client has to solve. Currently ``text'' for Textcaptchas and ``image'' for Imagecaptchas are supported.
\end{itemize}


Methods:

\begin{itemize} 
\item \verb|create(remote_ip, session_type)|: Responsible for basic creation of a \emph{CaptchaSession} of the requested type for the given IP adress. Only used for supercalls in the \verb|create()|-method of subclasses.
\end{itemize}


\paragraph{Attributes and Methods implemented in the subclass} \mbox{} \\


Attributes:


Each session needs to store the tokens, that were used for creating the session and additional information, that is needed for validating the answer given by the client. This can differ for every captchatype. 

TextCaptchaSession:

\begin{itemize} 
\item \verb|solved_captcha_token|: \emph{TextCaptchaToken}, that is allready solved and is used as a control word for the session.
\item \verb|unsolved_captcha_token|: \emph{TextCaptchaToken}, that is not solved and shall be solved by the client.
\item \verb|order|: Boolean indicating the order, in which the two tokens are displayed to the client.(0 -> solved unsolved 1 -> unsolved solved) It is needed to map the answers given by the client to the right tokens.
\end{itemize}


ImageCaptchaSession:

\begin{itemize}
\item \verb|image_token_list|: List of \emph{ImageCaptchaTokens}, where all tokens used for the session are saved.
\item \verb|order|: List of Booleans, that indicates which token in the \verb|image_token_list| is solved. (0 -> unsolved, 1-> solved)
\item \verb|task|: String, taht saves the task for the \emph{ImageCaptchaSession}, e.g. which objects should be detected in the images.
\end{itemize}

Methods: 

\begin{itemize}
\item \verb|create(remote_ip)|: Responsible for creating a \emph{CaptchaSession} and returning the created session to the corresponding \verb|view|.
\item \verb|validate(parameters)|: Responsible for validating the solution for a CaptchaSession and returning the created session to the corresponding \verb|view|. The solution suggested by the client is included in the parameters. Returns whether the session is valid or not.
\item \verb|renew()|: Responsible for exchanging the \emph{CaptchaTokens} of a \emph{CaptchaSession}, to create a new challenge or the same session.
\end{itemize}


\clearpage

	\section{Image Distortion}
\label{sec:image_distortion}

The fact that images for text Captchas are provided by users makes it impossible to tell if those images are easy to recognize for bots and are therefore safe to be used as Captcha token. In order to complicate the recognition of the Captcha token the systems uses an image distortion algorithm which is automatically applied to all uploaded text Captcha tokens.\\
The image distortion algorithm consists of two steps: the drawing of a horizontal line and a wave transformation. \\
In the first step it places a horizontal line in the middle of the image which is colored with the dominant color of the whole picture. Afterwards this line will be transformed together with the rest of the image.\\
The frequency as well as the amplitude of the wave which will be applied to text are dependent to the height of the image. Furthermore the frequency depends on the width of the image so that one wavelength is at least as wide as the image itself. In addition to this both, the frequency and the amplitude, are modified by a random value in order make every transformation unique. \\
Everything that is shifted out of bounds will be cut off. Additionally the pixels which were located at the bottom and the top of the the original picture will be stretched out vertically to fill the space which was emptied due to the transformation.\\

	\section{Implementation}
\label{sec:implementation}

\subsection{Server Side}
\label{subsec:Server Side}

Lorem ipsum dolor sit amet, consectetur adipiscing elit. In erat mauris, faucibus quis pharetra sit amet, pretium ac libero. Etiam vehicula eleifend bibendum. Morbi gravida metus ut sapien condimentum sodales mollis augue sodales. Vestibulum quis quam at sem placerat aliquet. Curabitur a felis at sapien ullamcorper fermentum. Mauris molestie arcu et lectus iaculis sit amet eleifend eros posuere. Fusce nec porta orci.

\subsection{Client Side}
\label{subsec:Client Side}
The Captcha service integrates seamlessly into existing web applications. It basically works by appending an invisible overlay to the body element of the existing website. The overlay fades in when the user hits the submit button of the captcha protected form. Thereby it does not affect the layout of the existing websites.
\begin{figure}[H]
	\centering
	\includegraphics[width=0.8\linewidth]{content/figures/captcha_words.png}
	\caption{Captcha card contents}
	\label{fig:captcha_words}
\end{figure}

 The overlay consists of a \textit{Captcha card}, which in turn consists of a \textit{task}, to be solved \textit{Captcha tokens} and \textit{submit} and \textit{refresh} action elements, as shown in fig. \ref{fig:captcha_words}. As mentioned earlier, we currently support two different Captcha types - namely \textit{text} and \textit{image captchas} - which are randomly delivered to the client. 
 
 \begin{figure}[H]
	\centering
	\includegraphics[width=0.8\linewidth]{content/figures/captcha_images.png}
	\caption{Image captcha session}
	\label{fig:captcha_images}
\end{figure}
 
 As soon as the user visits the page, the browser opens a session at the Captcha service via REST API. The browser receives a \textit{session key}, the \textit{Captcha type}, a list of \textit{Captcha tokens} and - in case of an \textit{image Captcha session} - a \textit{task}. The \textit{session key} gets stored in a hidden input field in the Captcha protected form. As a result the \textit{session key} is sent to the web application as soon as the form is submitted, in order to make sure that the client properly solved the Captcha at a later point in time. 
 
 Determined by the \textit{Captcha type} the \textit{Captcha tokens} get either rendered as \textit{text Captcha tokens}, as shown in fig. \ref{fig:captcha_words}, or as \textit{image Captcha grid}, as depicted in fig.\ref{fig:captcha_images}. In case the \textit{session type} is an \textit{image Captcha session}, the task is generated dynamically by inserting the delivered \textit{task} into the HTML, unlike the text Captcha task, which is static.
 
 When the client hits the reload button, the Captcha service is asked via REST API to renew the \textit{session's Captcha tokens}. The response consists of a new list of \textit{tokens} of the same type as before. Subsequently, the old \textit{Captcha tokens} are replaced by the freshly received \textit{Captcha tokens}. The \textit{submit action} triggers the server-side solution validation via REST API. In case the solution was correct the form gets submitted including the \textit{session key}. When the solution is incorrect, the \textit{Captcha card} shakes in order to visually indicate the wrong solution and all newly assigned \textit{Captcha tokens} are inserted. This happens in order to prevent brute force solving. As soon as a wrong solution is submitted, the Captcha service assigns new \textit{Captcha tokens} to the \textit{session} and returns them as response to the client. As a result, you can not restore the old \textit{session tokens} and try to solve it by using all possible inputs.

\subsection{Captcha Integration}
\label{subsec:Captcha Integration}

Integrating the Captcha service into your existing web application is fairly easy. You have to take the following steps:
\begin{enumerate}
	\item inject \texttt{captcha.min.css} and \texttt{captcha.min.js} to your HTML skeleton
	\item  Make sure the to be protected captcha form has the class \textit{captcha-form} and the corresponding submit button possesses the class \textit{captcha-button}
	\item Finally, integrate an additional POST request to our captcha service including the clients \textit{session key} into your existing server-side code, that receives the form data, in order to assure that the user solved the Captcha correctly and did not modify the javascript code.
\end{enumerate}

\texttt{captcha.min.js} includes the client-side code, whose behaviour is described in the last section. \texttt{captcha.min.css} provides the initial styling. Feel free to modify it, such that it fits your design. 

\clearpage

	\section{Solving Algorithm}
\label{sec:solving_algorithm}

In order to provide a value for the researchers which add data to the Captcha service, the system has to label the uploaded images. This becomes possible due to the solving algorithm, which determines the label based on the given user inputs. \\
In case of text Captchas the algorithm needs at least three users which solved the Captcha correctly. If three or more suggestions match, the image is marked as solved and labeled accordingly. However the token is identified as unsolvable if there are six or more proposals but no more than two of them match. This approach relatively similar to the concept reCAPTCHA uses. In a paper \footnote{http://science.sciencemag.org/content/321/5895/1465.full} that was published it was stated, that in most cases three human resolutions are enough to label the image reliably. \\
The method for labeling image Captchas is similar to the one used for texts. The main difference is the fact that the proposals for these are limited to \textit{true} and \textit{false}, are they suiting the specified task or not. Therefore the algorithm checks if at least four resolutions match and also declares a token as unsolvable if more the six suggestion are given but failed to produces four that match. It was decided to raise the bar for labeling a picture from three to four, because it is more likely to falsely select an image due to a wrong click.
	\section{Web Interface}
\label{sec:web_interface}

The web interface is a simple Django app, where researchers and scientists can register, upload and download images for the Captcha service to use.

\subsection{Upload}

The upload functionality allows for data to be uploaded for labeling or to feed tasks with already labeled data. The upload of pre-labeled data is necessary to guarantee the functionality of Captchas, thus the validation of a Captcha challenge.
First of all, the user chooses for which Captcha type he wants to upload his data and whether the data is already labeled or not.
For image Captchas, the user has to choose between an already existing task or he can add a task himself. The task declares which images are to be selected when solving the specific Captcha challenge.

For unsolved Captchas, the user simply has to upload a .zip-file with a folder containing his images. The .zip-file and the included folder need to have the same name.
For solved Captchas, the .zip-file must also contain a .txt-file in CSV format. The .txt-file needs to contain each image name, followed with \verb|True| or \verb|False| for image Captchas, or their solution word for text Captchas, separated by a comma.

The restrictions for the .txt-file are currently very strict. As a result, if the .txt-file contains an entry without a corresponding image in the image folder or an empty line between entries, the upload will fail. If one of the images submitted is not included in the .txt-file, it will be ignored.
\begin{figure}[H]
\centering
\includegraphics[width=1\linewidth]{content/figures/upload.png}
\caption{Upload page from web interface}
\label{fig:upload}
\end{figure}

\clearpage
\subsubsection{Directory structure for upload}

\textbf{Text Captcha (Unsolved):}

In the .zip-file there has to be an directory, which has the same name as the .zip-file. All captcha images you wish to upload have to be directly located within the directory.
\\
\dirtree{%
.1 directory-name.zip.
.2 directory-name.
.3 text-captcha1.png.
.3 text-captcha2.png.
.3 text-captcha3.png.
.3  ....
}

\textbf{Text Captcha (Solved):}

The directory tree is the same as for unsolved text Captchas. However, there is also an .txt-file, which contains the solutions for text Captcha.
\\
\dirtree{%
.1 directory-name.zip.
.2 random-name.txt.
.2 directory-name.
.3 text-captcha1.png.
.3 text-captcha2.png.
.3 text-captcha3.png.
.3  ....
}

The .txt-file has the following structure with a semicolon as a delimiter between text Captcha and its solution. The .txt-file may not contain any empty lines, as well as filenames which do not exist in the image-directory.

\begin{lstlisting}
text-captcha1.png; solution1
text-captcha1.png; solution2
text-captcha3.png; solution3
text-captcha4.png; solution4
text-captcha5.png; solution5
text-captcha6.png; solution6
\end{lstlisting}

Note that all images in the directory, which do not occur in the .txt-file are ignored and not uploaded!

\clearpage
\textbf{Image Captcha (Unsolved):}

The .zip-file must contain a directory with the same name as the .zip-file, containing all images. See \textbf{Text Captcha (Unolved)} as reference.
\\
\dirtree{%
.1 directory-name.zip.
.2 directory-name.
.3 image-captcha1.png.
.3 image-captcha2.png.
.3 image-captcha3.png.
.3  ....
}

\textbf{Image Captcha (Solved):}
\\
\dirtree{%
.1 directory-name.zip.
.2 random-name.txt.
.2 directory-name.
.3 image-captcha1.png.
.3 image-captcha2.png.
.3 image-captcha3.png.
.3  ....
}

The .txt-file has the following structure with a semicolon as a delimiter between image Captcha and True or False. True or False indicates if the image belongs to the selected task. The .txt-file may not contain any empty lines, as well as filenames which do not exist in the image-directory.

\begin{lstlisting}
text-captcha1.png; True
text-captcha1.png; True
text-captcha3.png; True
text-captcha4.png; True
text-captcha5.png; False
text-captcha6.png; False
\end{lstlisting}

Note that all images in the directory, which do not occur in the .txt-file are ignored and not uploaded!

\clearpage
\subsubsection{Possible errors during upload}

\clearpage
\subsection{Download}

The download functionality allows for data to be downloaded by choosing the Captcha type. The researcher can specify his task for image Captchas, whereas that is not possible for text Captchas. By pressing the download button, all Captchas matching the chosen type and/or task are downloaded into a .zip-file, together with a .txt-file. For text Captchas, the .txt-file contains all images names followed with their labeled word. For image Captchas, the .txt-file contains all images names followed with \verb|True| or \verb|False|, whether they match their specific task.
\begin{figure}[H]
\centering
\includegraphics[width=1\linewidth]{content/figures/download.png}
\caption{Download page from web interface}
\label{fig:download}
\end{figure}

\subsection{User authentication and registration}

The user authentication and registration is built using the Django authentication system \footnote{https://docs.djangoproject.com/en/1.10/topics/auth/default/}. Users can register, whereby an Email-address is needed in order to minimize Spam. The authentication Email is sent from an Email-address, which is configured in settings.py.
The user receives an activation link, which is active for 7 days. After clicking the activation link, the user is able to login into the web interface.
\begin{figure}[H]
\centering
\includegraphics[width=0.8\linewidth]{content/figures/email_settings.png}
\caption{Email settings in settings.py}
\label{fig:email_settings}
\end{figure}

In order for the activation link to work, the site has to be configured within the Django administration. The site has to match the server address, where the Captcha service runs.
\begin{figure}[H]
\centering
\includegraphics[width=1\linewidth]{content/figures/site_administration.png}
\caption{Configure site to match server address in the Django administration}
\label{fig:site_administration}
\end{figure}

\clearpage

	\section{Evaluation}
\label{sec:evaluation}

As proven by reCAPTCHA, the solving algorithm is accurate most of the time. However, there remains a small chance that images are labeled incorrectly. Certainly there is no way to completely eliminate the possibility of inaccurate labeled images. It would be possible to minimise this probability by further increasing the number of needed proposals, but this would slow the labeling process down and therefore result in fewer labeled images over time. The current implementation provides an acceptable trade-off in order to offer satisfying results for researchers, who rely on this service for data labeling.

Another major feature of the CaptchaService is the ability to reliably combat bots and spammers. The security features, which are in place right now will be enough to hold off the majority of all attacks. However one could argue, that it is not sufficient to renew the Captcha after a wrong proposal and validate the solved session of a user. Further security mechanism, such as blocking of ip addresses after consecutive inaccurate solutions could further increase the security of the system. Nevertheless, it will also worsen the user experience of clients that fail to solve Captchas reliably. This could ultimately lead to users not visiting the website again. Therefore the current implementation does not rely on any further security mechanisms and thus offers an optimal trade-off between user experience an security.

\clearpage

	\section{Future Work}
\label{sec:future_work}

With the thought in mind of building an easy expandable service, the logic consequence would be on focusing on different Captcha types. Another important Captcha type would be the labeling within one image. Therefore, an uploaded image could be split up into nine smaller images and like in our already implemented image Captcha, the user has to select the images with a specific object in it. In the process, key factors such as access for disabled users can be tackled, e.g. by implementing audio Captchas. The accessibility for all users is important to allow for an unrestricted usage of the web services using the Captcha service.

Another aspect would be expanding the web interface. The option of downloading solved and unsolved Captchas should be further able to specialize, e.g. by selecting specific upload times or certain time spans. Since newly uploaded data might extend existing tasks, a researcher might only want to extract his uploaded Captchas.
The option for uploading images could need some improvements concerning it's usability, since their are strict requirements for the structure and naming of the uploaded .zip-files. More flexibility in choice about how to upload the data and error feedback would simplify the usage of the captcha service.


Integrating feedback of the labeling progress within a task or uploaded data, is another feature necessary to improve the Captcha service.

\clearpage

	%%%%%%%%
	
	%%% BIBLIOGRAPHY
	%\bibliographystyle{babunsrt3-fl}
	\addcontentsline{toc}{section}{Bibliography}
	\bibliographystyle{babunsrt-fl}
	\bibliography{projektbib}
	\clearpage	

\end{document}

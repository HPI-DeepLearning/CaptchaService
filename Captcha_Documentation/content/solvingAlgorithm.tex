\section{Solving Algorithm}
\label{sec:solving_algorithm}

In order to provide a value for the researchers which add data to the Captcha service, the system has to label the uploaded images. This becomes possible due to the solving algorithm, which determines the label based on the given user inputs. \\
In case of text Captchas the algorithm needs at least three users which solved the Captcha correctly. If three or more suggestions match, the image is marked as solved and labeled accordingly. However the token is identified as unsolvable if there are six or more proposals but no more than two of them match. This approach relatively similar to the concept reCAPTCHA uses. In a paper \footnote{http://science.sciencemag.org/content/321/5895/1465.full} that was published it was stated, that in most cases three human resolutions are enough to label the image reliably. \\
The method for labeling image Captchas is similar to the one used for texts. The main difference is the fact that the proposals for these are limited to \textit{true} and \textit{false}, are they suiting the specified task or not. Therefore the algorithm checks if at least four resolutions match and also declares a token as unsolvable if more the six suggestion are given but failed to produces four that match. It was decided to raise the bar for labeling a picture from three to four, because it is more likely to falsely select an image due to a wrong click.